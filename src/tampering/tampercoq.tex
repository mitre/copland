\documentclass[12pt]{report}
\usepackage[T1]{fontenc}
\usepackage{lmodern}
\usepackage{fullpage}
\usepackage{coqdoc}
\usepackage{makeidx}
\usepackage{amsmath,amssymb}
\usepackage{url}
\usepackage{hyperref}

\iffalse
\title{Copland Attestation Terms:\\
  Semantics and Coq Proofs\thanks{Approved for Public Release;
    Distribution Unlimited. Public Release Case Number 18-3576.}}

\author{John D. Ramsdell\qquad\qquad Paul D. Rowe\\[3ex]
  The MITRE Corporation}
\fi

\makeindex

\begin{document}

\begin{titlepage}
  \vspace*{7ex}
  \begin{center}\LARGE
    Evidence Tampering and Chain of Custody\\
    in Layered Attestations Proofs
  \end{center}
  \vspace{3ex}
  \begin{center}\Large
    Ian D. Kretz\qquad Clare C. Parran\\
    John D. Ramsdell \qquad Paul D. Rowe\\[3ex]
    The MITRE Corporation\\[3ex]
    July 2023
  \end{center}
  \vfill The view, opinions, and/or findings contained in this report
  are those of The MITRE Corporation and should not be construed as an
  official Government position, policy, or decision, unless designated
  by other documentation.\\ \copyright 2023 The MITRE Corporation.
  Distributed under a standard, three-clause BSD license included as
  \texttt{license.txt}.  This software was produced for the
  U. S. Government under Basic Contract No. W56KGU-18-D-0004, and is
  subject to the Rights in Noncommercial Computer Software and
  Noncommercial Computer Software Documentation Clause 252.227-7014
  (FEB 2014). Approved for Public Release; Distribution
  Unlimited. Public Release Case Number 23-2487.
\end{titlepage}

%\maketitle

\begin{abstract}
  This document is a companion to the paper ``Evidence Tampering and
  Chain of Custody in Layered Attestations''.  It includes Coq
  translations of the definitions, theorems, and lemmas in the paper,
  along with proofs of all of the theorems and lemmas.  The Coq proof
  assistant ensures that all proof steps are valid using the logic of
  the Calculus of Inductive Constructions.  The abstract from the
  paper follows.

  In distributed systems, trust decisions are made on the basis of
  integrity evidence generated via remote attestation. Examples of the
  kinds of evidence that might be collected are boot time image hash
  values; fingerprints of initialization files for userspace
  applications; and a comprehensive measurement of a running kernel.
  In layered attestations, evidence is typically composed of
  measurements of key subcomponents taken from different trust
  boundaries within a target system. Discrete measurement evidence is
  bundled together for appraisal by the components that collectively
  perform the attestation.

  In this paper, we initiate the study of evidence chain of custody
  for remote attestation.  Using the Copland attestation specification
  language, we formally define the conditions under which a runtime
  adversary active on the target system can tamper with measurement
  evidence.  We present algorithms for identifying all such tampering
  opportunites for given evidence as well as tampering ``strategies''
  by which an adversary can modify incriminating evidence without
  being detected.  We then define a procedure for transforming a
  Copland-specified attestation into a maximally tamper-resistant
  version of itself.  Our efforts are intended to help attestation
  protocol designers ensure their protocols reduce evidence tampering
  opportunities to the smallest, most trustworthy set of components
  possible.
\end{abstract}

\tableofcontents

\coqlibrary{Introduction}{}{Introduction}
  The paper ``Evidence Tampering and Chain of Custody in Layered
  Attestations'' studies the chain of custody of measurements taken
  while performing layered attestations.  It provides carefully
  reasoned descriptions of how and when an adversary can tamper with
  collected evidence, and ways of reducing tampering opportunities.
  The paper includes precise mathematical description of its
  assertions including definitions, theorems, and lemmas.  None of the
  proofs of the theorems and lemmas are included in the paper.

  This document is a companion to the above paper.  It includes Coq
  translations of the definitions, theorems, and lemmas in the paper,
  along with proofs of all of the theorems and lemmas.

  The document is organized as is a traditional paper on
  mathematics---items are defined before being used.  Each item from
  the paper has an entry in the index that is included at the end of
  this paper.

\input{body}

\printindex

\end{document}
